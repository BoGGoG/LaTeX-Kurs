\RequirePackage[l2tabu, orthodox]{nag}
\documentclass[ngerman, hyperref={pdfpagelabels=false}]{beamer}
%\documentclass[handout]{beamer}
\usepackage[ngerman]{babel}
\usepackage{amsmath}
\usepackage{amsfonts}
\usepackage{amssymb}
\usefonttheme{serif}
\usepackage[utf8]{inputenc}
\usepackage {graphicx}
\usepackage [T1]{fontenc}
%\usepackage{listings} %Quellcode mit \lstinline{} oder mit \begin ... darstellen
\usepackage{hyperref}
\usepackage{fancyvrb}

\usepackage{siunitx}
\usepackage{cancel}
\usepackage{slashed}
\usepackage{braket}

\usepackage{cprotect}

\usepackage{animate}
\usepackage{media9}
\usepackage{spreadtab}

%\usepackage{caption}
%\usepackage{subcaption}
%\captionsetup[subfigure]{labelformat=empty}

% TikZ
%\usepackage[landscape]{geometry}
\usepackage{adjustbox}

\usepackage{tikz}
\usetikzlibrary{arrows}
\usetikzlibrary{calendar,shadings}
%\usetikzlibrary{mindmap}
%\usepackage{metalogo}
%\usepackage{dtklogos}
%\usepackage{adjustbox}

\usetikzlibrary{decorations.markings}
\tikzstyle{vertex}=[circle, draw, inner sep=0pt, minimum size=13pt]
\newcommand{\vertex}{\node[vertex]}
\usepackage{calc}
\newcounter{Angle}

\colorlet{winter}{blue}
\colorlet{spring}{green!60!black}
\colorlet{summer}{orange}
\colorlet{fall}{red}
\renewcommand*{\familydefault}{\sfdefault}
\newcount\mycount

\usepackage{bbding}
\usepackage{overpic}

\usepackage{cleveref}
%\crefname{equation}{Glg.}{Glgn.}

%\pgfpagesuselayout{4 on 1 with notes}[a4paper,border shrink=5mm]
\title[TiKz]{TikZ - Das Package mit dem Backronym als Namen \\
\flqq TikZ ist kein Zeichenprogramm\frqq}
\institute{Fachschaft Physik}
\author[Ole, Marco]{Ole Hinrichs, Marco Knipfer}
%\usetheme{Warsaw}
\usetheme{Madrid}
\usecolortheme{default}
\beamertemplatenavigationsymbolsempty
\providecommand{\thispdfpagelabel}[1]{}

\begin{document} 
\begin{frame}{}
    \maketitle
\end{frame}

\begin{frame}{Ein kleines Beispiel vorneweg}
    Beispiel von Kjell Magne Fauske 
    \url{http://www.texample.net/tikz/examples/global-nodes/}
     %For every picture that defines or uses external nodes, you'll have to
% apply the 'remember picture' style. To avoid some typing, we'll apply
% the style to all pictures.
    \tikzstyle{every picture}+=[remember picture]

% By default all math in TikZ nodes are set in inline mode. Change this to
% displaystyle so that we don't get small fractions.
    \everymath{\displaystyle}

    \begin{itemize}
        \item Coriolis acceleration
            \tikz\node [fill=blue!20,draw,circle] (n1) {};
    \end{itemize}

% Below we mix an ordinary equation with TikZ nodes. Note that we have to
% adjust the baseline of the nodes to get proper alignment with the rest of
% the equation.
    \begin{equation}
        \vec{a}_p = \vec{a}_o+\frac{{}^bd^2}{dt^2}\vec{r} +
        \tikz[baseline]{
            \node[fill=blue!20,anchor=base] (t1)
            {$ 2\vec{\omega}_{ib}\times\frac{{}^bd}{dt}\vec{r}$};
        } +
        \tikz[baseline]{
            \node[fill=red!20,anchor=base] (t2)
            {$\vec{\alpha}_{ib}\times\vec{r}$};
        } +
        \tikz[baseline]{
            \node[fill=green!20,anchor=base] (t3)
            {$\vec{\omega}_{ib}\times(\vec{\omega}_{ib}\times\vec{r})$};
        }
    \end{equation}

    \begin{itemize}
        \item Transversal acceleration
            \tikz\node [fill=red!20,draw,circle] (n2) {};
        \item Centripetal acceleration
            \tikz\node [fill=green!20,draw,circle] (n3) {};
    \end{itemize}

% Now it's time to draw some edges between the global nodes. Note that we
% have to apply the 'overlay' style.
    \begin{tikzpicture}[overlay]
        \path[->] (n1) edge [bend left] (t1);
        \path[->] (n2) edge [bend right] (t2);
        \path[->] (n3) edge [out=0, in=-90] (t3);
    \end{tikzpicture}
\end{frame}
\begin{frame}{Noch ein Beispiel}
   Authors: Till Tantau, Stefan Kottwitz\\
   \url{http://www.texample.net/tikz/examples/calendar-circles/}
   \begin{adjustbox}{max totalsize={.9\textwidth}{.7\textheight},center}
   \begin{tikzpicture}[transform shape,
       every day/.style={anchor=mid,font=\tiny}]
       \node[circle,shading=radial,outer color=blue!30,inner color=white,
       minimum width=15cm] {\textcolor{blue!80!black}{\Huge\the\year}};
       \foreach \month/\monthcolor in
       {1/winter,2/winter,3/spring,4/spring,5/spring,6/summer,
           7/summer,8/summer,9/fall,10/fall,11/fall,12/winter} {
      % Computer angle:
           \mycount=\month
           \advance\mycount by -1
           \multiply\mycount by 30
           \advance\mycount by -90
           \shadedraw[shading=radial,outer color=\monthcolor!30,middle color=white,
           inner color=white,draw=none] (\the\mycount:5.4cm) circle(1.4cm);
      % The actual calendar
           \calendar at (\the\mycount:5.4cm) [
           dates=\the\year-\month-01 to \the\year-\month-last]
           if (day of month=1) {\large\color{\monthcolor!50!black}\tikzmonthcode}
           if (Sunday) [red]
           if (all) {
      % Again, compute angle
               \mycount=1
               \advance\mycount by -\pgfcalendarcurrentday
               \multiply\mycount by 11
               \advance\mycount by 90
       \pgftransformshift{\pgfpointpolar{\mycount}{1.2cm}}};}
   \end{tikzpicture}
   \end{adjustbox}
\end{frame}
\begin{frame}[fragile]{Ok, wie funktioniert dieses TikZ?}{Ein TikZ Hello World}
    Basierend auf 
    \begin{footnotesize}
        \url{http://www.viitoriolimpici.ro/uploads/attach_data/38/9/6//an_interactive_introduction_to_latex_part_3.pdf}
    \end{foonotesize}
    \\ ~\\
    \begin{minipage}{.4\textwidth}
        \begin{tikzpicture}
            \draw (0,0) -- (1,1); 
        \end{tikzpicture}
    \end{minipage}
    \cprotect\fbox{
    \begin{minipage}{.4\textwidth}
        \begin{Verbatim}
\begin{tikzpicture}
    \draw (0,0) -- (1,1); 
\end{tikzpicture}
        \end{Verbatim}
    \end{minipage}
}
\pause
\\ ~\\~\\
Hilfreich, das Grid anzeigen zu lassen:\\
    \begin{minipage}{.3\textwidth}
        \begin{tikzpicture}
            \draw[help lines] (0,0) grid (3,3);
            \draw (0,0) -- (1,1); 
        \end{tikzpicture}
    \end{minipage}
    \cprotect\fbox{
    \begin{minipage}{.66\textwidth}
        \begin{Verbatim}
\begin{tikzpicture}
    \draw[help lines] (0,0) grid (3,3);
    \draw (0,0) -- (1,1); 
\end{tikzpicture}
        \end{Verbatim}
    \end{minipage}
}
\end{frame}
\begin{frame}[fragile]{Weiter TikZ Grundlagen}
    \begin{itemize}
        \item Sachen werden mit \verb!\draw! gezeichnet
        \item Jeder Befehl endet mit \verb!;!
            \pause
        \item Pfeile?\\
            \begin{minipage}{.25\textwidth}
            \begin{tikzpicture}
                \draw[help lines] (0,0) grid (3,3);
                \draw[->] (0,0) -- (1,1);
                \draw[<->, thick] (2,1) -- (1,2);
                \draw[<-, thick, dashed] (2,2)--(3,3);
            \end{tikzpicture}
            \end{minipage}
            \begin{minipage}{.67\textwidth}
            \begin{Verbatim}[fontsize=\small, frame=single]
\begin{tikzpicture}
    \draw[help lines] (0,0) grid (3,3);
    \draw[->] (0,0) -- (1,1);
    \draw[<->, thick] (2,1) -- (1,2);
    \draw[<-, thick, dashed] (2,2)--(3,3);
\end{tikzpicture}
            \end{Verbatim}
        \end{minipage}
        \pause
    \item Pfade, indem man einfach mehrere Punkte angibt:\\
        \verb!\draw[<->, thick] (0,4)--(0,0)--(4,0);!
    \end{itemize}
\end{frame}
\begin{frame}[fragile]{Pfade}
    \begin{minipage}{0.3\textwidth}
    \begin{tikzpicture}
        \draw[help lines] (0,0) grid (3,4);
        \draw[<->, thick] (0,4)--(0,0)--(3,0); 
        \draw (.5,.5) -- (.5,3.5) --
        (2.5,3.5) -- (2.5,.5) --
        cycle;
    \end{tikzpicture}
    \end{minipage}
    \begin{minipage}{0.55\textwidth}
        \begin{Verbatim}
\begin{tikzpicture}
    \draw[help lines] (0,0) grid (3,4);
    \draw[<->, thick] (0,4)--(0,0)--(3,0); 
    \draw (.5,.5) -- (.5,3.5) --
    (2.5,3.5) -- (2.5,.5) --
    cycle;
\end{tikzpicture}
        \end{Verbatim}
    \end{minipage}
\end{frame}
\begin{frame}[fragile]{Ausfüllen von Flächen}
    \begin{minipage}{0.27\textwidth}
    \begin{tikzpicture}
        \draw[help lines] (0,0) grid (3,4);
        \draw[<->, thick, red] (0,4)--(0,0)--(3,0); 
        \draw[blue, fill=yellow] (.5,.5) -- (.5,3.5) --
        (2.5,3.5) -- (2.5,.5) --
        cycle;
    \end{tikzpicture}
    \end{minipage}
    \begin{minipage}{0.55\textwidth}
        \begin{Verbatim}[fontsize=\small]
\begin{tikzpicture}
 \draw[help lines] (0,0) grid (3,4);
 \draw[<->, thick, red] (0,4)--(0,0)--(3,0); 
 \draw[blue, fill=yellow (.5,.5) -- (.5,3.5) --
 (2.5,3.5) -- (2.5,.5) --
 cycle;
\end{tikzpicture}
        \end{Verbatim}
    \end{minipage}
\end{frame}
\begin{frame}[fragile]{einfache Formen: Kreis, Rechteck}
    \begin{minipage}{0.27\textwidth}
    \begin{tikzpicture}
        \draw[help lines] (0,0) grid (3,4);
        \draw[fill=blue] (1,1) circle (0.4);
        % Mittelpunkt, Radius
        \draw[thick, blue] (1,2) rectangle (2.5, 3);
        % links unten, rechts oben
    \end{tikzpicture}
    \end{minipage}
    \begin{minipage}{0.55\textwidth}
        \begin{Verbatim}[fontsize=\small]
\begin{tikzpicture}
    \draw[help lines] (0,0) grid (3,4);
    \draw[fill=blue] (1,1) circle (0.4);
    % Mittelpunkt, Radius
    \draw[thick, blue] (1,2) rectangle (2.5, 3);
    % links unten, rechts oben
\end{tikzpicture}
        \end{Verbatim}
    \end{minipage}
\end{frame}

\begin{frame}[fragile]{Nodes - Praktische Koordinaten}
    \begin{minipage}{0.27\textwidth}
    \begin{tikzpicture}
        \draw[help lines] (0,0) grid (3,4);
        \node (a) at (.3, .3) {a};
        \node (b) at (1.5, 2) {$\phi$};
        \node (c) at (.3, 2.5) {$\xi$};
        \node (d) at (2.2, .3) {$\zeta$};
        \draw[->] (a) -- (b);
        \draw[->] (a) -- (c);
        \draw[->] (a) -- (d);
    \end{tikzpicture}
    \end{minipage}
    \begin{minipage}{0.55\textwidth}
        \begin{Verbatim}[fontsize=\small]
    \begin{tikzpicture}
        \draw[help lines] (0,0) grid (3,4);
        \node (a) at (.3, .3) {a};
        \node (b) at (1.5, 2) {$\phi$};
        \node (c) at (.3, 2.5) {$\xi$};
        \node (d) at (2.2, .3) {$\zeta$};
        \draw[->] (a) -- (b);
        \draw[->] (a) -- (c);
        \draw[->] (a) -- (d);
    \end{tikzpicture}
        \end{Verbatim}
    \end{minipage}
\end{frame}
\begin{frame}[fragile]{Funktionen Plotten}
    \begin{minipage}{0.27\textwidth}
    \begin{tikzpicture}
        \draw[help lines] (0,-1) grid (3,4);
        \draw[->] (0,0) -- (0,3);
        \draw[->] (0,0) -- (3,0);
        \node (a) at (0, 0) {$0$};
        \node (b) at (2, 2) {$x$};
        \draw[->] (a) -- (b);
        \draw[blue,domain=0:3]
            plot (\x, {sin(\x r)});
        \draw[red,domain=0:3]
        plot (\x, {\x - .166*\x^3});
    \end{tikzpicture}
    \end{minipage}
    \begin{minipage}{0.55\textwidth}
        \begin{Verbatim}[fontsize=\small]
\begin{tikzpicture}
    \draw[help lines] (0,-1) grid (3,4);
    \draw[->] (0,0) -- (0,3);
    \draw[->] (0,0) -- (3,0);
    \node (a) at (0, 0) {$0$};
    \node (b) at (2, 2) {$x$};
    \draw[->] (a) -- (b);
    \draw[blue,domain=0:3]
        plot (\x, {sin(\x r)});
    \draw[red,domain=0:3]
        plot (\x, {\x - .166*\x^3});
\end{tikzpicture}
        \end{Verbatim}
    \end{minipage}
\end{frame}
\begin{frame}{Wie war das mit dem Pagerank-Algorithmus?}
    \begin{minipage}{.39\textwidth}
        \begin{tikzpicture}[x=1.3cm, y=1cm,
                every edge/.style={
                    draw,
                    postaction={decorate,
                        decoration={markings,mark=at position 0.5 with {\arrow{>}}}
                    }
                }
            ]
            \vertex (A) at (0,0) [label=above: A]{1};
            \vertex (B) at (1,1) [label=above: B]{1};
            \vertex (C) at (1,-1) [label=above: C]{1};
            \vertex (D) at (2,0) [label=above: D]{1};
            \vertex (E) at (3,1) [label=above: E]{1};
            \path
            (A) edge (B)
            (A) edge (C)
            (B) edge (D)
            (C) edge (D)
            (E) edge (B)
            ;
        \end{tikzpicture}
    \end{minipage}
    \begin{minipage}{.55\textwidth}
        \begin{equation*}
            \mathrm{pr}_{n+1} = \frac{1-d}{|V|} + d \cdot \sum_{\substack{w\in V
            \text{mit}\\w\to v}} \frac{\mathrm{pr}_{n}(w)}{\mathrm{deg}_{\text{out}(w)}}\,.
            \label{eq:pagerank}
        \end{equation*}
    \end{minipage}
    \\ ~\\
$d = 0.85$; $|V| = 5$. 
\begin{equation*}
   \begin{array}{c | lllll}
       n\backslash \pr & \pr(A) & \pr(B) & \pr(C) & \pr(D) & \pr(E) \\
       \hline
       0 & 1 & 1 & 1 & 1 & 1 \\
       1 & 0.03 & 1.305 & 0.455 & 1.73 & 0.03 \\
       2 & 0.03 & 0.06825 & 0.04275 & 1.526 & 0.03 \\
       3 & 0.03 & 0.06825 & 0.04275 & 0.12435 & 0.03 
   \end{array} 
\end{equation*}
\end{frame}
\begin{frame}[fragile]{Der Graph}
    \begin{minipage}{0.3\textwidth}
        \begin{tikzpicture}
            \node[vertex] (A) at (0,0) [label=above: A]{1};
            \node[vertex] (B) at (1,1) [label=above: B]{1};
            \node[vertex] (C) at (1,-1) [label=above: C]{1};
            \node[vertex] (D) at (2,0) [label=above: D]{1};
            \node[vertex] (E) at (3,1) [label=above: E]{1};
            \draw[->] (A) -- (B); 
            \draw[->] (A) -- (C); 
            \draw[->] (B) -- (D); 
            \draw[->] (C) -- (D); 
            \draw[->] (E) -- (B); 
        \end{tikzpicture}
    \end{minipage}
    \begin{minipage}{0.6\textwidth}
        \begin{Verbatim}[textsize=\tiny]
\begin{tikzpicture}
    \node[vertex] (A) at (0,0) 
                [label=above: A]{1};
    \node[vertex] (B) at (1,1) 
                [label=above: B]{1};
    \node[vertex] (C) at (1,-1) 
                [label=above: C]{1};
    \node[vertex] (D) at (2,0) 
                [label=above: D]{1};
    \node[vertex] (E) at (3,1) 
                [label=above: E]{1};
    \draw[->] (A) -- (B); 
    \draw[->] (A) -- (C); 
    \draw[->] (B) -- (D); 
    \draw[->] (C) -- (D); 
    \draw[->] (E) -- (B); 
\end{tikzpicture}
        \end{Verbatim}
    \end{minipage}
\end{frame}
\begin{frame}[fragile]{Graph mit Nodes proportional zu Wichtigkeit}
    \begin{adjustbox}{max totalsize={.9\textwidth}{.7\textheight},center}
        \begin{tikzpicture}
    % minimum size = 40 * pr(Vertex)
            \vertex (A) at (0,0) [minimum size = 1.2cm, label=above: A]{0.03};
            \vertex (B) at (2,3) [minimum size = 2.73cm, label=above: B]{0.06825};
            \vertex (C) at (2,-3) [minimum size = 1.71cm, label=above: C]{0.04275};
            \vertex (D) at (5,0) [minimum size = 4.974cm, label=above: D]{0.12435};
            \vertex (E) at (8,3) [minimum size = 1.2cm, label=above: E]{0.03};
            \draw[->] (A) -- (B); 
            \draw[->] (A) -- (C); 
            \draw[->] (B) -- (D); 
            \draw[->] (C) -- (D); 
            \draw[->] (E) -- (B); 
            ;
        \end{tikzpicture}
    \end{adjustbox}
    \begin{Verbatim}[textsize=\tiny]
\node[vertex] (A) at (0,0) 
    [minimum size = 1.2cm, label=above: A]{0.03};
    \end{Verbatim}
\end{frame}
\end{document}
